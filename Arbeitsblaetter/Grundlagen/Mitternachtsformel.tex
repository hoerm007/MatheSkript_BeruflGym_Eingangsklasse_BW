% !TeX root = ../../Skript.tex
\cohead{\Large\textbf{Mitternachtsformel}}
\fakesubsection{Mitternachtsformel}
Die Mitternachtsformel oder \(abc\)-Formel zum Berechnen der Lösungen einer quadratischen Gleichung ist eine der wichtigsten Lösungsformeln.
\begin{tcolorbox}
	Für eine Gleichung der Form
	\begin{align*}
		\textcolor{loestc}{ax^2+bx+c=0,\ a\neq 0}
	\end{align*}
	können die Lösungen wie folgt bestimmt werden:
	\begin{align*}
		\textcolor{loestc}{x_{1/2}=\frac{-b\pm \sqrt{b^2-4ac}}{2a}}
	\end{align*}
\end{tcolorbox}
Abhängig von der Diskriminante \(\left( b^2-4ac\right) \) hat eine quadratische Gleichung entweder 2 Lösungen (Diskriminante positiv), 1 Lösung (Diskriminante ist 0) oder keine Lösung (Diskriminante negativ).

\textbf{Beispiel:}
\begin{enumerate}[label=\arabic*)]
	\item 2 Lösungen:
	\begin{align*}
		2x^2+6x-8&=0\\
		\textcolor{loes}{x_{1/2}}&\textcolor{loes}{\;=\frac{-6\pm \sqrt{6^2-4\cdot2\cdot(-8)}}{2\cdot2}=\frac{-6\pm \sqrt{100}}{4}=\frac{-6\pm 10}{4}}\\
		\textcolor{loes}{\Rightarrow x_1}&\textcolor{loes}{\;=1 \text{ und } x_2=-4}
	\end{align*}

    \bigskip

	\item 1 Lösung:
	\begin{align*}
		\frac{1}{2}x^2-2x+2&=0\\
		\textcolor{loes}{x_{1/2}}&\textcolor{loes}{\;=\frac{-(-2)\pm \sqrt{(-2)^2-4\cdot\frac{1}{2}\cdot2}}{2\cdot\frac{1}{2}}=2\pm\sqrt{0}=2}
	\end{align*}

    \bigskip

	\item Keine Lösungen:
	\begin{align*}
		-x^2+2x-3&=0\\
	\textcolor{loes}{x_{1/2}}&\textcolor{loes}{\;=\frac{-2\pm \sqrt{2^2-4\cdot(-1)\cdot(-3)}}{2\cdot(-1)}=\frac{-2\pm\sqrt{-8}}{-2}\Lightning}
	\end{align*}\\
	\textcolor{loes}{Das Blitzsymbol {\Lightning}\ zeigt an, dass die Gleichung keine Lösung hat, weil unter der Wurzel nie eine negative Zahl stehen darf (zumindest in den reellen Zahlen).}
\end{enumerate}

\newpage
%%%%%%%%%%%%%%%%%%%%%%%%%%%%%%%%%%%%%%%%%%%%%%%%%%%%%%%%%%%%%%%%%%%%%
\begin{Exercise}[title={Löse die folgenden Gleichungen}, label=mitternachtA1]

	\begin{minipage}{\textwidth}
		\begin{minipage}{0.5\textwidth}
			\begin{enumerate}[label=\alph*)]
				\item \(x^2-3x+2=0\)
				\item \(x^2+5x+6=0\)
				\item \(x^2+2x-8=0\)
				\item \(2x^2-6x-8=0\)
				\item \(x^2-8x+16=0\)
				\item \(2x^2+x+1=0\)
				\item \(\frac{1}{4}x^2+\frac{1}{2}x-6=0\)
				\item \(4x^2+6x+\frac{9}{4}=0\)
				\item \(x^2+\frac{3}{2}x-1=0\)
				\item \(-2x^2+4x-3=0\)
				\item \(4x^2-11x+6=0\)
				\item \(x^2+\frac{1}{2}x-\frac{1}{2}=0\)
				\item \(-2x^2-6x-\frac{9}{4}=0\)
			\end{enumerate}
		\end{minipage}%
		\begin{minipage}{0.5\textwidth}
			\begin{enumerate}[label=\alph*)]
				\setcounter{enumi}{13}
				\item \(x-12+x^2=0\)
				\item \(4x^2-x-\frac{1}{2}=0\)
				\item \(-2x^2-6x-\frac{9}{2}=0\)
				\item \(3+2x+x^2=0\)
				\item \(-\frac{1}{4}x^2+x-2=0\)
				\item \(x^2-2x-2=0\)
				\item \(3x^2-3x-\frac{3}{2}=0\)
				\item \(-1x^2+x=0\)
				\item \(3x-2x^2-1=0\)
				\item \(3x^2-2x+1=0\)
				\item \(3x^2-4x+\frac{4}{3}=0\)
				\item \(1-4x^2+3x=0\)
				\item \(-2x^2+\frac{3}{2}x+\frac{1}{4}=0\)
			\end{enumerate}
		\end{minipage}%
	\end{minipage}%
\end{Exercise}
\newpage
%%%%%%%%%%%%%%%%%%%%%%%%%%%%%%%%%%%%%%%%%
\begin{Answer}[ref=mitternachtA1]

	\begin{minipage}{\textwidth}
		\begin{minipage}{0.46\textwidth}
			\begin{enumerate}[label=\alph*)]
				\item \(x_1=1,\ x_2=2\)
				\item \(x_1=-2,\ x_2=-3\)
				\item \(x_1=2,\ x_2=-4\)
				\item \(x_1=4,\ x_2=-1\)
				\item \(x_{1/2}=4\)
				\item keine Lösungen
				\item \(x_1=4,\ x_2=-6\)
				\item \(x_{1/2}=-\frac{3}{4}\)
				\item \(x_1=\frac{1}{2},\ x_2=-2\)
				\item keine Lösungen
				\item \(x_1=\frac{3}{4},\ x_2=2\)
				\item \(x_1=\frac{1}{2},\ x_2=-1\)
				\item \(x_{1/2}=-\frac{3}{2}\)
			\end{enumerate}
		\end{minipage}%
		\begin{minipage}{0.54\textwidth}
			\begin{enumerate}[label=\alph*)]
				\setcounter{enumi}{13}
				\item \(x_1=3,\ x_2=-4\)
				\item \(x_1=\frac{1}{2},\ x_2=-\frac{1}{4}\)
				\item \(x_{1/2}=-\frac{3}{2}\)
				\item keine Lösungen
				\item keine Lösungen
				\item \(x_1=1+\sqrt{3},\ x_2=1-\sqrt{3}\)
				\item \(x_1=\frac{1+\sqrt{3}}{2},\ x_2=\frac{1-\sqrt{3}}{2}\)
				\item \(x_1=\frac{-1+\sqrt{5}}{2},\ x_2=\frac{-1-\sqrt{5}}{2}\)
				\item \(x_1=1,\ x_2=\frac{1}{2}\)
				\item keine Lösungen
				\item \(x_{1/2}=\frac{2}{3}\)
				\item \(x_1=-\frac{1}{4},\ x_2=1\)
				\item \(x_1=\frac{3+\sqrt{17}}{8},\ x_2=\frac{3-\sqrt{17}}{8}\)
			\end{enumerate}
		\end{minipage}%
	\end{minipage}%
\end{Answer}