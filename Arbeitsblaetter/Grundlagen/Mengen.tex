% !TeX root = ../../Skript.tex
\cohead{\Large\textbf{Mengen}}
\fakesubsection{Mengen}
Die folgenden mathematischen Grundkenntnisse sind unabdingbare Voraussetzung zum Verständnis der folgenden Kapitel.

Die Mengenlehre ist ein grundlegendes Teilgebiet der Mathematik. Wir werden uns hier nur mit den für uns relevanten Mengen beschäftigen:
\begin{tcolorbox}
	\textbf{Zahlenmengen}
	\begin{itemize}
		\item Die natürlichen Zahlen: \({\N=\textcolor{loestc}{\{(0), 1; 2; 3; 4; 5; 6; \ldots\}}}\)

        Die Mathematiker können sich nicht einigen, ob die 0 mit eingeschlossen sein soll. Daher wird meist \(\N^*\) für die natürlichen Zahlen ohne die 0 verwendet und \(\N_0\) für die natürlichen Zahlen mit der 0.
		\item Die ganzen Zahlen: \({\Z=\textcolor{loestc}{\{0; 1; -1; 2; -2; 3; -3; 4; -4;\ldots\}}}\)

        Ergänzt man die natürlichen Zahlen um das Vorzeichen, so erhält man die ganzen Zahlen.
		\item Die rationalen Zahlen: \({\Q=\textcolor{loestc}{\{\frac{n}{m}|n\in\Z, m\in\N^*\}}}\)

        Die rationalen Zahlen enthalten alle Zahlen, die sich als Brüche mit einem Zähler aus den ganzen Zahlen und einem Nenner aus den natürlichen Zahlen (natürlich ohne der 0) darstellen lassen.
		\item Die reellen Zahlen: \(\R\)

        In den reellen Zahlen \(\R\) liegen "`alle"' Zahlen, zumindest alle uns bekannten Zahlen. \(\R\) beinhaltet neben \(\Q\) auch Zahlen wie \(\sqrt{2}\) oder \(\pi\).
	\end{itemize}
\end{tcolorbox}
Mengen lassen sich auf verschiedene Arten darstellen. Nehmen wir als Beispiele die Menge aller positiven, geraden Zahlen \(G\) und die Menge \(H\) aller Zahlen, die größer oder gleich \(1\) und kleiner \(2\) sind:
\begin{itemize}
	\item Aufzählung: \({G=\{2; 4; 6; 8; 10;\ldots \}}\)
	\item Einschränkung einer übergeordneten Menge \({G=\{x\in\N|x\text{ ist gerade}\}}\) oder

	\({H=\{x\in\R|-1<=x<2\}}\)
	\item Darstellung als Intervall: \(H=[-1;2)\) Dabei steht die eckige Klammer für ein abgeschlossenes Ende, d.h. die Grenze liegt noch im Intervall und die runde Klammer für ein offenes Intervall, d.h. die Grenze liegt nicht mehr im Intervall.
\end{itemize}
Liegt eine Zahl in einer Menge, z.B. \(-2\) in \(\Q\), so schreibt man \({-2\in\Q}\) (Sprich \(-2\) ist Element der rationalen Zahlen).