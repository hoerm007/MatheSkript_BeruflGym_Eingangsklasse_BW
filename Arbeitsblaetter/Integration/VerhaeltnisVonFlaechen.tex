% !TeX root = ../../Skript.tex
\cohead{\Large\textbf{Verhältnis von Flächen}}
\fakesubsection{Verhältnis von Flächen}
\begin{tcolorbox}
	\textbf{Verhältnis von Flächen:}

	\textcolor{loestc}{Das Verhältnis zweier Flächen lässt sich bestimmen, indem man die beiden Flächen durcheinander teilt, z.B. stehen die beiden Flächen \(A_1=2\) und \(A_2=4\) im Verhältnis \(1:2\), da \(\dfrac{A_1}{A_2}=\dfrac{1}{2}\) gilt.
	}
\end{tcolorbox}
Beispiele:

\bigskip

\begin{minipage}{\textwidth}
	\adjustbox{valign=t}{\begin{minipage}{.6\textwidth}\raggedright
		Zeige, dass die Gerade \textcolor{blue}{\(g(x)=-x\)} die von der \(x\)-Achse und der Funktion \textcolor{red}{\(f(x)=x^2-4x\)} eingeschlossene Fläche \(A\) im Verhältnis \(27:37\) teilt:

		\textcolor{loes}{Die zwei Teilflächen lassen sich z.B. wie folgt bestimmen:}

		\textcolor{loes}{Fläche zwischen \(f(x)\) und der \(x\)-Achse:}.

		\textcolor{loes}{\(\displaystyle A=-\int\limits_0^4 x^2-4x \td x=\frac{32}{3}\)}

		\textcolor{loes}{Fläche zwischen \(f(x)\) und \(g(x)\):}.

		\textcolor{loes}{\(\displaystyle A_1=\int\limits_0^3 -x-\left(x^2-4x\right) \td x=\frac{9}{2}\)}\

		\textcolor{loes}{Die Fläche, die von der \(x\)-Achse und den beiden Funktionen eingeschlossen wird, beträgt also \(A_2=A-A_1=\tfrac{37}{6}\)}.

		\textcolor{loes}{Damit teilt die Gerade die Fläche \(A\) im Verhältnis \(\dfrac{A_1}{A_2}=\dfrac{27}{37}\)}.
	\end{minipage}}%
	\adjustbox{valign=t, padding=3ex 0ex 0ex 0ex}{\begin{minipage}{.4\textwidth-3ex}
		\includegraphics[width=\textwidth]{\integration/pics/VerhaeltnisBsp1\iftoggle{ausfuellen}{}{_empty}.png}
	\end{minipage}}%
\end{minipage}

\vspace{2cm}

\begin{minipage}{\textwidth}
	\adjustbox{valign=t, padding=0ex 0ex 3ex 0ex}{\begin{minipage}{.4\textwidth-3ex}
		\includegraphics[width=\textwidth]{\integration/pics/VerhaeltnisBsp2\iftoggle{ausfuellen}{}{_empty}.png}
	\end{minipage}}%
	\adjustbox{valign=t}{\begin{minipage}{.6\textwidth}\raggedright
		Bestimme in welchem Verhältnis die \(y\)-Achse die von \textcolor{red}{\(f(x)=-1,5x^2+3x+4,5\)} und der \(x-\)Achse eingeschlossenen Fläche teilt.

		\textcolor{loes}{Die zwei Teilflächen lassen sich z.B. wie folgt bestimmen:}

		\textcolor{loes}{Flächeninhalt der linken der beiden Teilflächen:}.

		\textcolor{loes}{\(\displaystyle A_1=-\int\limits_{-1}^0 -1,5x^2+3x+4,5 \td x=\frac{5}{2}\)}

		\textcolor{loes}{Flächeninhalt der rechten der beiden Teilflächen:}.

		\textcolor{loes}{\(\displaystyle A_2=-\int\limits_0^3 -1,5x^2+3x+4,5 \td x=\frac{27}{2}\)}

		\textcolor{loes}{Damit teilt die \(y\)-Achse die Fläche zwischen der Funktion und der \(x\)-Achse im Verhältnis \(\dfrac{A_1}{A_2}=\dfrac{5}{27}\)}.
	\end{minipage}}%
\end{minipage}
\newpage

\begin{Exercise}[title={\raggedright\normalfont Gegeben sind die Gerade \(x=1\) und die Funktion \(f(x)=-\frac{1}{2}x^2+3x\), ihr Schaubild sei \(K_f\) Die Fläche \(A\) sei die von \(K_f\) und der \(x\)-Achse eingeschlossene Fläche.}, label=verhaltnisFlaechenA1]
	\begin{enumerate}[label=\alph*)]
		\item Zeichne \(K_f\) und die Gerade \(x=1\) in ein Koordinatensystem.
		\item Zeige, dass die Gerade \(x=1\) die Fläche \(A\) im Verhältnis 2:25 teilt.
	\end{enumerate}
\end{Exercise}

\begin{Exercise}[title={\raggedright\normalfont Gegeben sind die Funktion \(g(x)=\frac{1}{6}x^3+\frac{1}{2}x^2\) und ihr Schaubild \(K_g\).}, label=verhaltnisFlaechenA2]

	\begin{minipage}{\textwidth}
		\adjustbox{valign=t, padding=0ex 0ex 3ex 0ex}{\begin{minipage}{.6\textwidth-3ex}
			\includegraphics[width=\textwidth]{\integration/pics/VerhaeltnisAuf2.png}
		\end{minipage}}%
		\adjustbox{valign=t}{\begin{minipage}{.4\textwidth}\raggedright
			\begin{enumerate}[label=\alph*)]
				\item \(K_g\), die \(x\)-Achse und die Gerade \(x=-4\) schließen 2 Flächen ein. Zeige, dass diese gleich groß sind.
				\item \(K_g\), die \(x\)-Achse und die Gerade \(x=2\) schließen 2 Flächen ein. Zeige, dass diese im Verhältnis 16:9 stehen.
			\end{enumerate}
		\end{minipage}}%
	\end{minipage}
\end{Exercise}

\begin{Exercise}[title={\raggedright\normalfont Gegeben sind die Funktion \(h(x)=-\frac{1}{18}x^3+\frac{1}{2}x^2\) und ihr Schaubild \(K_h\).}, label=verhaltnisFlaechenA3]

    \begin{minipage}{\textwidth}
    	\adjustbox{valign=t}{\begin{minipage}{.4\textwidth}\raggedright
    			\begin{enumerate}[label=\alph*)]
    				\item Zeige, dass die von \(K_h\) und der \(x\)-Achse eingeschlossene Fläche \(A\) den Flächeninhalt \(\frac{243}{8}\) hat.
    				\item Zeichne die Gerade \(i(x)=-\frac{1}{2}x+\frac{9}{2}\) ein. Der Schnittpunkt der Geraden \(i(x)\) mit \(K_h\) ist \(S(3\vert 3)\).

    				Zeige, dass die Gerade die Fläche \(A\) im Verhältnis 16:11 teilt.
    			\end{enumerate}
    		\end{minipage}}%
    		\adjustbox{valign=t, padding=3ex 0ex 0ex 0ex}{\begin{minipage}{.6\textwidth-3ex}
    			\includegraphics[width=\textwidth]{\integration/pics/VerhaeltnisAuf3.png}
    		\end{minipage}}%
	\end{minipage}
\end{Exercise}

%%%%%%%%%%%%%%%%%%%%%%%%%%%%%%%%%%%%%%%%%
\begin{Answer}[ref=verhaltnisFlaechenA1]

	\begin{minipage}{\textwidth}
		\adjustbox{valign=t, padding=0ex 0ex 3ex 0ex}{\begin{minipage}{.4\textwidth-3ex}
			\includegraphics[width=\textwidth]{\integration/pics/VerhaeltnisAuf1.png}
		\end{minipage}}%
		\adjustbox{valign=t}{\begin{minipage}{.6\textwidth}\raggedright
			Die zwei Teilflächen lassen sich z.B. wie folgt bestimmen:

			Flächeninhalt der linken der beiden Teilflächen:.

			\(\displaystyle A_1=\int\limits_{0}^1 -\frac{1}{2}x^2+3x \td x=\frac{4}{3}\)

			Flächeninhalt der rechten der beiden Teilflächen:.

			\(\displaystyle A_2=-\int\limits_1^6 -\frac{1}{2}x^2+3x \td x=\frac{50}{3}\)

			Damit teilt die Gerade \(x=1\) die Fläche zwischen der Funktion und der \(x\)-Achse im Verhältnis \(\dfrac{A_1}{A_2}=\dfrac{2}{25}\).
		\end{minipage}}%
	\end{minipage}
\end{Answer}
\begin{Answer}[ref=verhaltnisFlaechenA2]

	\begin{enumerate}[label=\alph*)]
		\item Berechnen der beiden Flächen:
		\begin{align*}
			-\int\limits_{-4}^{-3}\frac{1}{6}x^3+\frac{1}{2}x^2 \td x&=\frac{9}{8}\\
			\int\limits_{-3}^{0}\frac{1}{6}x^3+\frac{1}{2}x^2 \td x&=\frac{9}{8}
		\end{align*}
		Die beiden Flächen haben also beide den gleichen Flächeninhalt von \(\frac{9}{8}\).
		\item Berechnen der beiden Flächen:
		\begin{align*}
			\int\limits_{-3}^{0}\frac{1}{6}x^3+\frac{1}{2}x^2 \td x&=\frac{9}{8}\\
			\int\limits_{0}^{2}\frac{1}{6}x^3+\frac{1}{2}x^2 \td x&=2
		\end{align*}
		Die beiden Flächen stehen also im Verhältnis \(\dfrac{2}{9/8}=\dfrac{16}{9}\).
	\end{enumerate}
\end{Answer}
\begin{Answer}[ref=verhaltnisFlaechenA3]
	\begin{enumerate}[label=\alph*)]
		\item Berechnen die Fläche \(A\):
		\[A=\int\limits_0^9 -\frac{1}{18}x^3+\frac{1}{2}x^2 \td x=\frac{243}{8}\]
		\item Berechnen der beiden Flächen:
		\[A_2=\int\limits_3^9 -\frac{1}{18}x^3+\frac{1}{2}x^2-\left(-\frac{1}{2}x+\frac{9}{2}\right) \td x=18\]
		Die Fläche \(A_1\) ergibt sich dann aus
		\[A_1=A-A_2=\frac{243}{8}-18=\frac{99}{8}\]
		Die Gerade teilt die Fläche \(A\) also im Verhältnis \(\dfrac{A_2}{A_1}=\dfrac{18}{99/8}=\dfrac{16}{11}\)
    \end{enumerate}

    \bigskip

    \adjustbox{valign=t}{\includegraphics[width=\textwidth]{\integration/pics/VerhaeltnisAuf3_Loesung.png}}
\end{Answer}
